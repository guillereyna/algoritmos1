\documentclass[spanish, a4paper]{article}
\setlength{\parindent}{0em}
\usepackage{a4wide}
\usepackage{amsmath, amscd, amssymb, amsthm, latexsym}
\usepackage[spanish,activeacute]{babel}
\usepackage{enumerate}
\usepackage[utf8]{inputenc}
\usepackage{amsmath}
\newcommand{\tupla}[2]{#1 \times #2}
\input{Algo1Macros}

\setlength{\parskip}{0.1em}
\usepackage{caratula} % Version modificada para usar las macros de algo1 de ~> https://github.com/bcardiff/dc-tex



\begin{document}
\newcommand{\toroide}{\textit{toroide}}
\newcommand{\bin}{\ensuremath{\ent \times \ent}}

\titulo{TP de Especificación}
\fecha{\today}
\materia{Algoritmos y Estructuras de Datos I}
\grupo{Grupo: INSERTE NÚMERO AQUÍ}

% CAMBIAR INTEGRANTES
\integrante{Wazowski, Mike}{002/12}{mwazowski@dc.uba.ar}
\integrante{Bond, James}{007/007}{bond@mi6.co.uk}
\integrante{Capunta, Elsa}{003/17}{ecapunta@dc.uba.ar}
\integrante{Sánchez, Warren}{004/15}{tienelasrespuestas@les.luthier.com}

\maketitle



\section{Ejercicios - Primera Parte}



%ESCRIBIR  SOLUCIONES AQUÍ

\begin{ejercicio}[:]
\textbf{pred esValido(t: $\toroide$)}


\end{ejercicio}

\begin{ejercicio}[:]
\textbf{pred toroideMuerto(t: $\toroide$)}

\end{ejercicio}


\begin{ejercicio}[:]
\textbf{pred posicionesVivas(t: $\toroide$, vivas : $\TLista{\tupla{\ent}{\ent}}$)}

\end{ejercicio}

\begin{ejercicio}[:]
\textbf{aux densidadPoblacion(t: $\toroide$) = $\float$}

\end{ejercicio}

\begin{ejercicio}[:]
\textbf{aux cantVecinosVivos(t: $\toroide$, f: $\ent$, c: $\ent$) = $\ent$}

\end{ejercicio}


\begin{ejercicio}[:]
\textbf{pred evolucionDePosicion(t: $\toroide$, posicion : $\tupla{\ent}{\ent}$)}

\end{ejercicio}


\begin{ejercicio}[:]
\textbf{pred evolucionToroide(t1: $\toroide$, t2: $\toroide$)}

\end{ejercicio}





\section{Ejercicios - Segunda Parte}

%COMANDOS UTILES:
	% \hspace{4cm}: deja un espacio de 4cm entre cada línea 
	% \\ comienza una nueva línea 	

\begin{ejercicio}[:] \textbf{proc evolucionMultiple(in t: $\toroide$, in k: $\ent$, out result: $\toroide$)}\\

\begin{proc}{evolucionMultiple}{in t: $\toroide$, in k: $\ent$, out result: $\toroide$}{}
    \pre{(k > 0) \land esValido(t)} % k = 0 sería valido?
    \post{
    esLaKEsimaEvolucion(result, t, k)
    }

\end{proc}

\begin{pred}{esLaKesimaEvolucion}{evolucion: \toroide, t: \toroide, k: \ent}{\\
    \text{/* Decide si $evolucion$ es la k-ésima evolución de $t$ */}
    \\
    (\exists l : \TLista{\toroide})(\\ \\
    \longitud{l} \geq k+1 \land \\
    toroidesValidos(l) \yLuego \\(
    l[0] = t \land \\
    dimensionesIguales(l) \land\\
    (\forall i : \ent)(0 \leq i < \longitud{l} - 1) \implicaLuego evolucionToroide(l[i], l[i+1])) \\ \yLuego \\
    result = l[k]) \\
}

\end{pred}

\begin{pred}{dimensionesIguales}{ts: \TLista{\toroide}}{\\
    \text{/* Dada una secuencia de toroides, decide si todos los toroides tienen la misma dimensión */} \\
    (\forall i,j: \ent)((0 \leq i,j < \longitud{ts}) \implicaLuego dim[ts[i]] = dim[ts[j]]) \\
}
    
\end{pred}
\begin{aux}{dim}{t: \toroide}{$\ent \times \ent$}{(\longitud{t}, \longitud{t[0]})}
\end{aux}

\end{ejercicio}


\begin{ejercicio}[:] \textbf{proc esPeriodico(in t: $\toroide$, inout p: $\ent$, out result: $\bool$)}\\

\begin{proc}{esPeriodico}{in t: $\toroide$, inout p: $\ent$, out result: $\bool$}{}
    \pre{(p = P_0) \land esValido(t)}
    \post{\\
    \text{/* Es periódico si existe algún k $>$ 0 tal que la k-ésima evolución de t es t */}\\
    result = \True \Iff esPeriodo(t) \yLuego \\
    ((result = \True \Then !(\exists k : \ent)(0 < k < p \yLuego esLaKEsimaEvolucion(t, t, k)) \land \\
    (result = \False \Then p = P_0))\\
    }

\end{proc}

\begin{pred}{esPeriodo}{t: toroide}{\\
(\exists k : \ent)(k > 0 \yLuego esLaKEsimaEvolucion(t, t, k))\\
}
\end{pred}

\end{ejercicio}


\begin{ejercicio}[:] \textbf{proc primosLejanos(in t1: $\toroide$, in t2: $\toroide$, out primos: $\bool$)}
\\

\begin{proc}{primosLejanos}{in t1: $\toroide$, in t2: $\toroide$, out primos: $\bool$}{}
    \pre{(esValido(t1) \land esValido(t2)) \yLuego dim(t1) = dim(t2)}
    \post{ \\
    primos = \True \Iff (\exists k : \ent)(k > 0 \yLuego \\
    (esLaKEsimaEvolucion(t2, t1, k) \lor esLaKEsimaEvolucion(t1, t2, k))) \\
    }

\end{proc}

\end{ejercicio}



\begin{ejercicio}[:] \textbf{proc seleccionNatural(in ts: $\TLista{\toroide}$, out res: $\ent$)}\\

\begin{proc}{seleccionNatural}{in ts: $\TLista{\toroide}$, out res: $\ent$}{}
    \pre{\\
    (\forall i : \ent)(0 \leq i < \longitud{ts} \implicaLuego esValido(ts[i])) \yLuego \\
    (dimensionesIguales(ts) \land \\
    todasMuerenEnAlgunTick(ts)) \\
    }
    \post{\\
    res \in ts \land \\
    !(\exists t : \toroide)(t \in ts \land muereAntesQue(res, t))\\
    }

\end{proc}

\begin{pred}{todasMuerenEnAlgunTick}{ts: \TLista{\toroide}} { \\
    (\forall t: \toroide)(t \in ts \implicaLuego \\
    (\exists k: \ent)(k > 0 \yLuego muereEnKEsimoTick(t, k))) \\
    }
\end{pred}

\begin{pred}{muereAntesQue}{antes: \toroide, despues: \toroide}{\\
    (\exists n,m : \ent)((n,m>0) \land (n < m) \\
    \land muereEnKEsimoTick(antes, n) \land muereEnKEsimoTick(despues, m) \\
}

\begin{pred}{muereEnKEsimoTick}{t: \toroide, k: \ent}{ \\
    (\exists m: \toroide)(dim(m) = dim(t) \yLuego toroideMuerto(m) \yLuego esLaKEsimaEvolucion(m, t, k)) \land \\
    !(\exists i : \ent)( (0 < i < k) \yLuego esLaKEsimaEvolucion(t, m, k))
}

\end{pred}

\end{pred}

\end{ejercicio}


\begin{ejercicio}[:] \textbf{proc fusionar(in t1: $\toroide$, in t2: $\toroide$, out res: $\toroide$)}
\\

\begin{proc}{fusionar}{in t1: $\toroide$, in t2: $\toroide$, out res: $\toroide$}{}
    \pre{(esValido(t1) \land esValido(t2)) \yLuego dim(t1) = dim(t2)}
    \post{\\
    esValido(res) \yLuego (\\
    dim(res) = dim(t1) \yLuego \\
    (\forall i, j : \ent)((0 \leq i < \longitud{t1}) \land (0 \leq j < \longitud{t1[0]}) \implicaLuego \\
    (res[i][j] = \True \Iff t1[i][j] = \True \land t2[i][j] = \True)) \\
    }

\end{proc}

\end{ejercicio}


\begin{ejercicio}[:] \textbf{proc vistaTrasladada(in t1: $\toroide$, in t2: $\toroide$, out res: $\bool$)}\\

\begin{proc}{vistaTrasladada}{in t1: $\toroide$, in t2: $\toroide$, out res: $\bool$}{}
    \pre{(esValido(t1) \land esValido(t2)) \yLuego dim(t1) = dim(t2)}
    \post{res=\True \Iff esRotacion(t1, t2)}

\end{proc}

\begin{pred}{esRotacion}{t1: \toroide, t2: \toroide}{ \\
    dim(t1) = dim(t2) \yLuego \\
    (\exists k,l :\ent)( \\
    (\forall i,j :\ent)((0 \leq i < \longitud{t1} \land 0 \leq j < \longitud{t1[0]}) \implicaLuego \\
    t1[i][j] = t2[(i+k)\% \longitud{t2}][(j+l) \% \longitud{t2[0]}]))  \text{/* Para que no se me pase de rango */}\\
}

\end{pred}

\end{ejercicio}



\begin{ejercicio}[:] \textbf{proc menorSuperficieViva(in t: $\toroide$, out res: $\ent$)}
\\

\begin{proc}{menorSuperficieViva}{in t: $\toroide$, out res: $\ent$}{}
    \pre{esValido(t) \land !toroideMuerto(t)}
    \post{
    esAreaDeMenorSuperficieViva(res, t)
    }

\end{proc}

\begin{pred}{esAreaDeMenorSuperficieViva}{n: \ent, t: \toroide}{\\
    existeTraslacionConSuperficieVivaDeArea(n, t) \yLuego \\
    noExisteTraslacionConAreaMenorViva(n, t) \\
}
    
\end{pred}

\begin{pred}{existeTraslacionConSuperficieVivaDeArea}{n: \ent, t: \toroide}{\\
(\exists t0 : \TLista{\toroide})(esValido(t0) \yLuego esRotacion(t, t0) \yLuego \\
existeSuperficieVivaDeArea(n, t0) \\
}
\end{pred}

\begin{pred}{existeSuperficieVivaDeArea}{n: \ent, t: \toroide}{\\
\text{/* Acá dejo de considerarlo un toroide, pues estoy mirando para cada traslación. */} \\
\text{/* Sería mejor no mirar cada traslación y considerarlo un toroide?? */} \\
(\exists i,j : \ent)((0 \leq i < \longitud{t} \land 0 \leq j < \longitud{t[0]}) \yLuego \\ 
(\forall k,l : \ent)(0 \leq k \leq (\longitud{t}-i) \land 0 \leq l \leq (\longitud{t[0]} - j) \implicaLuego \\ (sumaVivasRango(t, i, j, k, l) = posicionesVivas(t) \land n = $(k+1-i)\cdot(l+1-j)$)) \\
}
\end{pred}

\begin{aux}{sumaVivasRango}{t: \toroide, desde_x: \ent, desde_y: \ent, hasta_x: \ent, hasta_y: \ent}{\ent}{\\
$\sum_{i=0}^{\longitud{t}-1} \sum_{j=0}^{\longitud{t[0]-1}} \IfThenElse{(desde_x \leq i < hasta_x) \land (desde_y \leq j < hasta_y) \land t[i][j] = \True}{1}{0}$
}

\end{aux}

\begin{pred}{noExisteTraslacionConAreaMenorViva}{n: \ent, t: \toroide}{\\
    !(\exists t0 : \TLista{\toroide})(esValido(t0) \yLuego esRotacion(t, t0) \yLuego \\
    (\exists m: \ent)(0 < m < n \yLuego existeSuperficieVivaDeArea(m, t0))) \\
}
\end{pred}

\end{ejercicio}


\begin{ejercicio}[:] \textbf{proc enCrecimiento(in t: $\toroide$, out res: $\bool$)}\\

\begin{proc}{enCrecimiento}{in t: $\toroide$, out res: $\bool$}{}
    \pre{esValido(t)}
    \post{
    res = true \Iff (\exists evol: \toroide)(esLaKesimaEvolucion(evol, t, 1) \yLuego suSuperficieVivaEsMayorQueLaDe(t, evol))
    }

\end{proc}

\begin{pred}{suSuperficieVivaEsMayorQueLaDe}{t_{min}: \toroide, t_{max}: \toroide}{\\
    (\exists s_{min}, s_{max})(esAreaDeMenorSuperficieViva(s_{min}, t_{min}) \land esAreaDeMenorSuperficieViva(s_{max}, t_{max}) \land \\
    s_{min} < s_{max})
}

\end{pred}

\end{ejercicio}


\section{Decisiones tomadas}

%Aquí deben agregar todo lo que hayan asumido para realizar cada ejercicio. No tienen que hablar sobre la estructura (por ejemplo justificar por qué separaron predicados/auxiliares o explicarlos). Tampoco repetir lo que dice el enunciado, pero SI cualquier asunción extra que realicen (en caso de que hagan alguna).

%Tomar en cuenta que no pueden asumir cualquier cosa, ya que podrían estar simplificando el ejercicio, consulten cualquier decisión que tomen! :)

\end{document}
