\documentclass[a4paper]{article}
\input{Algo1Macros}
\usepackage[utf8]{inputenc}
\usepackage{a4wide}
\usepackage{amsmath, amscd, amssymb, amsthm, latexsym}
\usepackage[spanish,activeacute]{babel}
\usepackage{enumerate}
\usepackage{parskip}
\usepackage{comment}

\usepackage{caratula} % Version modificada para usar las macros de algo1 de ~> https://github.com/bcardiff/dc-tex

\newcommand{\toroide}{Toroide}

\setlength{\parindent}{2em} %espaciado horizontal
\setlength{\parskip}{0.5em} %espaciado vertical


\begin{document}

\titulo{TP de Especificación}
\fecha{\today}
\materia{Algoritmos y Estructuras de Datos I}
\grupo{Grupo  3}

% CAMBIAR INTEGRANTES
\integrante{Fernández Spandau, Luciana}{368/20}{fernandezspandau@gmail.com}
\integrante{Reyna Maciel, Guillermo}{393/20}{guille.j.reyna@gmail.com}
\integrante{Casado Farall, Joaquín}{072/20}{joakinfarall@gmail.com}
\integrante{Companeetz, Ezequiel}{415/20}{echucompa@gmail.com}

\maketitle



\section{Ejercicio 1}

\subsection{Predicado principal}
\pred{esValido}{t : \toroide} { \\
(\longitud{t} \geq 3 \land esMatriz(t)) \yLuego \longitud{t[0]} \geq 3 \\
}

\subsection{Predicados auxiliares}

\pred{esMatriz}{m: \matriz{T}}{\\
(\forall i : \ent)(0 \leq i < \longitud{m} \implicaLuego (\longitud{m[i]} > 0 \wedge (\forall j : \ent)(0 \leq j < \longitud{m} \implicaLuego \longitud{m[i]} = \longitud{m[j]}))\\
}

\section{Ejercicio 2}

\pred{toroideMuerto}{t : \toroide}{\\
	(\forall i : \ent)((0 \leq i < |t|) \implicaLuego (\forall j : \ent) ((0 \leq j < |t[i]|) \implicaLuego (t[i][j] = \False)))
	\\ }

\section{Ejercicio 3}

\subsection{Función principal}

\pred{posicionesVivas}{t: \toroide, vivas: \TLista{\ent \times \ent}}{\\ 
 (\forall i : \ent)((\forall j : \ent)(0 \leq i < \longitud{t} \yLuego 0 \leq j < \longitud{t[i]})) \implicaLuego ( (i,j) \in vivas \Iff t[i][j] = \True) \land sinRepetidos(vivas) ) \\ 
}

\subsection{Predicados auxiliares}

\pred{sinRepetidos}{vivas : \TLista{\ent \times \ent }  }{ \\
 ( \forall i,j : \ent ) ( (0 \leq i,j < |vivas| \land i \neq j) \Rightarrow_L vivas[i] \neq vivas[j] ) \\
 }

\section{Ejercicio 4}

\subsection{Función principal}
\aux{densidadPoblacion}{t : toroide}{\float}{\\
	cantidadVivos(t) \ / \ (|t| \cdot |t[0]|)}

\subsection{Funciones auxiliares}
\aux{cantidadVivos}{t : toroide}{\float}{\\
	\sum_{i=0}^{|t|-1}
	\sum_{j=0}^{|t[i]|-1}
	\IfThenElse{t[i][j] = \True}{1}{0}}

\section{Ejercicio 5}

\subsection{Función principal}
\aux{cantVecinosVivos}{t : \toroide, f : \ent , c : \ent}{\ent}{\\
\sum_{i=0}^{|t|-1} \sum_{j=0}^{|t[i]| -1 } \IfThenElse{t[i][j] = \True \land sonVecinos(i,j,f,c,t)}{1}{0}
}

\subsection{Predicados auxiliares}
\pred{sonVecinos}{x,y : \ent , $v_1, v_2$ : \ent, t : \toroide}{\\
(v_1 \neq x \lor v_2 \neq y) \ \land \ (distancia(x, v_1, |t|) , \ distancia(y, v_2, |t[0]|) \leq 1) \\
} /*Siendo (x,y) las coordenadas de la posición a evaluar y $(v_1, v_2)$ las del vecino */

\subsection{Funciones auxiliares}
\aux{abs}{n : \float}{\float}{ \IfThenElse{n<0}{ -n}{n}}  
\aux{distancia}{$x_1$ : \ent, $x_2$ : \ent, rango: \ent}{\float}{abs (x_1- x_2) \ mod \  rango}
/*Siendo rango la variable para tener en cuenta los vecinos de los bordes*/

\section{Ejercicio 6}

\pred{evolucionDePosicion}{t : toroide, posicion : $\ent \times \ent$}{\\
	( t[(posicion)_0][(posicion)_1] = \True \land 
	2 \leq cantVecinosVivos(t, (posicion)_0, (posicion)_1) \leq 3 )\\ \lor 
	( t[(posicion)_0][(posicion)_1] = \False \land cantVecinosVivos(t, (posicion)_0, (posicion)_1) = 3 )
	\\}

\section{Ejercicio 7}

\pred{evolucionToroide}{t1: \toroide, t2: \toroide}{\\
(\longitud{t1} = \longitud{t2} \wedge \longitud{t1[0]} = \longitud{t2[0]}) \yLuego \\
(\forall i : \ent)((\forall j : \ent)((0 \leq i < \longitud{t1} \yLuego 0 \leq j < \longitud{t1[i]}) \implicaLuego \\
evolucionDePosicion(t1, (i,j))\Iff t2[i][j] = \True\\
}

\section{Ejercicio 8}

\subsection{Procedimiento}
\begin{proc}{evolucionMultiple}{\In t : toroide, \In k : \ent, \Out result : toroide}{}
    \pre{k > 0 \land esValido(t)}
    \post{esValido(result) \land kTicksEvolucion(k, t, result)}
\end{proc}

\subsection{Predicados auxiliares}
    \pred{kTicksEvolucion}{k : \ent, t, result : toroide}{\\
    (\exists evo : \TLista{toroide})((k = |evo| - 1 \land evo[0] = t \land evo[k] = result) \land_L \\
    (\forall i : \ent)(1 \leq i \leq k \implicaLuego (esValido(evo[i]) \land_L evolucionToroide(evo[i-1], evo[i]))))
    \\}

\section{Ejercicio 9}

\begin{proc}{esPeriodico}{\In t : toroide, \Inout p : \ent, \Out result : \bool}{}
    \pre{esValido(t)}
    \post{result = \True \leftrightarrow (p > 0 \ \land \ kTicksEvolucion(p, t, t) \land \\
    (\forall k : \ent)(k > 0 \land kTicksEvolucion(k,t,t) \implicaLuego k \leq p))}
\end{proc}

\section{Ejercicio 10}

\subsection{Procedimiento}
\begin{proc}{primosLejanos}{\In t1 : toroide, \In t2 : toroide, \Out primos : \bool}{}
    \pre{esValido(t1) \land esValido(t2) \land dim(t1, t2)}
    \post{result = \True \leftrightarrow sonPrimos(t1,t2)}
\end{proc}

\subsection{Predicados auxiliares}
\pred{dim}{t1, t2 : toroide}{\\
    |t1| = |t2| \land |t1[0]| = |t2[0]|
    \\}

\pred{sonPrimos}{t1, t2 : toroide}{\\
    (\exists k : \ent)(k > 0 \land (kTicksEvolucion(k, t1, t2) \lor kTicksEvolucion(k,t2,t1)))
    \\}

\section{Ejercicio 11}

\subsection{Procedimiento}
\begin{proc}{seleccionNatural}{\In ts : \TLista{toroide}, \Out res : \ent}{}
    \pre{sonValidos(ts)}
    \post{0 \leq res < |ts| \land maxTicks(res, ts)}
\end{proc}

\subsection{Predicados auxiliares}
\pred{sonValidos}{ts : \TLista{toroide}}{\\
    |ts| > 0 \land \\
    (\forall i : \ent)(0 \leq i < |ts| \implicaLuego esValido(ts[i]))
    \\}

\pred{maxTicks}{res : \ent, ts : \TLista{toroide}}{\\
    \neg muere(ts[res]) \lor_L (\forall i : \ent)(0 \leq i < |ts| \implicaLuego ticksVivo(ts[res]) \geq ticksVivo(ts[i]))
    \\}

\pred{muere}{t : toroide}{\\
    (\exists t2 : toroide)(sonPrimos(t, t2) \land toroideMuerto(t2))
    \\}

\subsection{Funciones auxiliares}
Esta función tendría que contar la cantidad de ticks que está vivo un toroide.
 Esto no funcionaría porque necesitaría un proceso que devuelva la evolución de un toroide
 para llamar, pero no se pueden llamar procesos dentro de funciones, ni usar iteración
 o recursividad. No se me ocurrió algo mejor. \\
 \\
\aux{ticksVivo}{t : toroide}{\ent}{funcion}

\section{Ejercicio 12}

\subsection{Procedimiento}
\begin{proc}{fusionar}{\In t1 : toroide, \In t2 : toroide, \Out res : toroide}{}
    \pre{esValido(t1) \land esValido(t2) \land mismoTablero(t1, t2)}
    \post{esValido(res) \land mismoTablero(t1,res) \land esInterseccion(t1, t2, res)}
\end{proc}

\subsection{Predicados auxiliares}
\pred{esInterseccion}{t1, t2, res : toroide}{\\
    (\forall i, j : \ent)((0 \leq i < |t1| \land 0 \leq j < |t1[0]) \implicaLuego
    (res[i][j] = \True \leftrightarrow (t1[i][j] = \True \land t2[i][j] = \True)))
    \\}

\section{Ejercicio 13}

\subsection{Procedimiento}
\begin{proc}{vistaTrasladada}{\In t1 : toroide, \In t2 : toroide, \Out res : \bool}{}
    \pre{esValido(t1) \land esValido(t2) \land mismoTablero(t1, t2)}
    \post{res = \True \leftrightarrow esTraslacion(t1, t2)}
\end{proc}

\subsection{Predicados auxiliares}
\pred{esTraslacion}{t1, t2 : toroide}{\\
    (\exists x, y : \ent)(\forall i, j : \ent)((0 \leq i < |t1| \land 0 \leq j < |t1[0]|) \implicaLuego \\
    t1[i][j] = t2[i + x \mod |t2|][j + y \mod |t2[0]|])
    \\}

\section{Ejercicio 14}

\subsection{Procedimiento}
\begin{proc}{menorSuperficieViva}{\In t : toroide, \Out res : \ent}{}
    \pre{esValido(t)}
    \post{res \geq 0 \land esMenorSuperficieEnTraslacion(t, res))}
\end{proc}

\subsection{Predicados auxiliares}
\pred{esMenorSuperficieEnTraslacion}{t : toroide, res : \ent}{\\
    (toroideMuerto(t) \land res = 0) \lor \\
    \neg (\exists t2 : toroide)(esTraslacion (t, t2) \land \\
    (\exists i1, i2, j1, j2)((0 \leq i1 < i2 < |t2| \land 0 \leq j1 < j2 < |t2[0]) \land_L \\
    (esMenorRectangulo(t2,i1,i2,j1,j2) \land area(i1,i2,j1,j2) < res)))
    \\}

\pred{esMenorRectangulo}{t : toroide, i1, i2, j1, j2 : \ent}{\\
    contieneTodosVivos(t, i1, i2, j1, j2) \land \\
    \neg (\exists i3, i4, j3, j4 : \ent)((0 \leq i3 < i4 < |t| \land 0 \leq j3 < j4 < |t[0]) \land_L \\
    contieneTodosVivos(t, i3, i4, j3, j4) \land area(i3, i4, j3, j4) < area(i1, i2, j1, j2))
    \\}

\pred{contieneTodosVivos}{t : toroide, i1, i2, j1, j2 : \ent}{\\
    cantidadVivos(t) = cantVivosEnRectangulo(t, i1, i2, j1, j2)
    \\}

\subsection{Funciones auxiliares}
\aux{area}{i1, i2, j1, j2 : \ent}{\ent}{\\
    (i2 - i1) * (j2 - j1)
}

\aux{cantVivosEnRectangulo}{t : toroide, i1, i2, j1, j2 : \ent}{\ent}{\\
    \sum_{i=i1}^{i2} \sum_{j=j1}^{j2} \IfThenElse{(t[i][j] = \True)}{1}{0} 
}

\section{Ejercicio 15}

\subsection{Procedimiento}
\begin{proc}{enCrecimiento}{\In t : toroide, \Out res : \bool}{}
    \pre{esValido(t)}
    \post{res = \True \leftrightarrow estaCreciendo(t)}
\end{proc}

\subsection{Predicados auxiliares}
\pred{estaCreciendo}{t : toroide}{\\
    (\exists t2 : toroide)(evolucionToroide(t, t2) \land \\
    (\exists k, m : \ent)((0 < k < m) \land_L (esMenorSuperficie(t, k) \land esMenorSuperficie(t2, m))))
    \\}

    
\section{Decisiones tomadas}
En cada ejercicio de la primera parte asumimos que el usuario debe cumplir con el contrato pasando un toroide válido, para simplificar la escritura.
 En general, en los predicados asumimos que los parámetros pasados son válidos cuando posible, y la validez de los parámetros pasados es verificada en los procedimientos.\\ 
\indent En el ejercicio 9, tomamos la decisión de que en caso de que el resultado sea falso, el parámetro p debe tener el mismo valor que se le pasó originalmente. \\
\indent En el ejercicio 10, decidimos que un toroide debe evolucionar una cantidad positiva de ticks. O sea, dado un toroide t, el mismo toroide t 
es primo de t únicamente si t es periódico. \\
\indent En el ejercicio 13, decidimos que la traslación (0, 0), que es igual a la traslación
($|t|$, $|t[0]$) es una traslación válida. 
También, nuestro procedimiento es indiferente a cuál traslación se está evaluando, siempre y cuando sea una traslación válida. 
Asumimos que, por ejemplo, una traslación de (-4, -3) es válida.\\
\indent Además de eso no tomamos ninguna decisión muy relevante al momento de resolver los ejercicios, más que interpretar los enunciados y a partir de eso escribir las especificaciones que pensamos adecuadas.
\end{document}

