\documentclass[a4paper]{article}
\input{Algo1Macros}
\usepackage[utf8]{inputenc}
\usepackage{a4wide}
\usepackage{amsmath, amscd, amssymb, amsthm, latexsym}
\usepackage[spanish,activeacute]{babel}
\usepackage{enumerate}
\usepackage{parskip}

\usepackage{caratula} % Version modificada para usar las macros de algo1 de ~> https://github.com/bcardiff/dc-tex

\newcommand{\toroide}{Toroide}

\setlength{\parindent}{2em} %espaciado horizontal
\setlength{\parskip}{0.5em} %espaciado vertical


\begin{document}

\titulo{TP de Especificación}
\fecha{\today}
\materia{Algoritmos y Estructuras de Datos I}
\grupo{Grupo  3}

% CAMBIAR INTEGRANTES
\integrante{Fernández Spandau, Luciana}{368/20}{fernandezspandau@gmail.com}
\integrante{Reyna Maciel, Guillermo}{393/20}{guille.j.reyna@gmail.com}
\integrante{Casado Farall, Joaquín}{072/20}{joakinfarall@gmail.com}
\integrante{Companeetz, Ezequiel}{415/20}{echucompa@gmail.com}

\maketitle

\section{Ejercicio 1}

\subsection{Predicado principal}
\pred{esValido}{t : \toroide} { \\
esMatriz(t) \land |t[0]| \geq 3 \land |t| \geq 3 \\
}

\subsection{Predicados auxiliares}
MATRIZ LULI

\section{Ejercicio 2}

\pred{toroideMuerto}{t : toroide}{\\
	(\forall i : \ent)((0 \leq i < |t|) \implicaLuego (\forall j : \ent) ((0 \leq j < |t[i]|) \implicaLuego (t[i][j] = \False)))
	\\ }

\section{Ejercicio 3}

PRED LULI \\
CHEQUEAR RANGO

\section{Ejercicio 4}

\subsection{Función principal}
\aux{densidadPoblacion}{t : toroide}{\float}{\\
	cantidadVivos(t) \ / \ (|t| \cdot |t[0]|)}

\subsection{Funciones auxiliares}
\aux{cantidadVivos}{t : toroide}{\float}{\\
	\sum_{i=0}^{|t|-1}
	\sum_{j=0}^{|t[i]|-1}
	\IfThenElse{t[i][j] = \True}{1}{0}}

\section{Ejercicio 5}

\subsection{Función principal}
\aux{cantVecinosVivos}{t : \toroide, f : \ent , c : \ent}{\ent}{\\
\sum_{i=0}^{|t|-1} \sum_{j=0}^{|t[i]| -1 } \IfThenElse{t[i][j] = \True \land sonVecinos(i,j,f,c,t)}{1}{0}
}

\subsection{Predicados auxiliares}
\pred{sonVecinos}{x,y : \ent , $v_1, v_2$ : \ent, t : \toroide}{\\
(v_1 \neq x \lor v_2 \neq y) \ \land \ (distancia(x, v_1, |t|) , \ distancia(y, v_2, |t[0]|) \leq 1) \\
} /*Siendo (x,y) las coordenadas de la posición a evaluar y $(v_1, v_2)$ las del vecino */

\subsection{Funciones auxiliares}
\aux{abs}{n : \float}{\float}{ \IfThenElse{n<0}{ -n}{n}}  
\aux{distancia}{$x_1$ : \ent, $x_2$ : \ent, rango: \ent}{\float}{abs (x_1- x_2) \ mod \  rango}
/*Siendo rango la variable para tener en cuenta los vecinos de los bordes*/

\section{Ejercicio 6}

\pred{evolucionDePosiciones}{t : toroide, posicion : $\ent \times \ent$}{\\
	\IfThenElse{t[(posicion)_0][(posicion)_1] = \True \\}
	{2 \leq cantVecinosVivos(t, (posicion)_0, (posicion)_1) \leq 3\\}
	{cantVecinosVivos(t, (posicion)_0, (posicion)_1) = 3}
	\\}

\section{Ejercicio 7}

COSO LULI

\section{Decisiones tomadas}
Asumimos en cada ejercicio que el toroide pasado es un toroide válido para simplificar la escritura.

\end{document}
