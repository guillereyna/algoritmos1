\documentclass[a4paper]{article}
\input{Algo1Macros}
\usepackage[utf8]{inputenc}
\usepackage{a4wide}
\usepackage{amsmath, amscd, amssymb, amsthm, latexsym}
\usepackage[spanish,activeacute]{babel}
\usepackage{enumerate}
\usepackage{parskip}
\usepackage{comment}

\usepackage{caratula} % Version modificada para usar las macros de algo1 de ~> https://github.com/bcardiff/dc-tex

\newcommand{\toroide}{Toroide}

\setlength{\parindent}{2em} %espaciado horizontal
\setlength{\parskip}{0.5em} %espaciado vertical


\begin{document}

\titulo{TP de Especificación}
\fecha{\today}
\materia{Algoritmos y Estructuras de Datos I}
\grupo{Grupo  3}

% CAMBIAR INTEGRANTES
\integrante{Fernández Spandau, Luciana}{368/20}{fernandezspandau@gmail.com}
\integrante{Reyna Maciel, Guillermo}{393/20}{guille.j.reyna@gmail.com}
\integrante{Casado Farall, Joaquín}{072/20}{joakinfarall@gmail.com}
\integrante{Companeetz, Ezequiel}{415/20}{echucompa@gmail.com}

\maketitle



\section{Ejercicio 1}

\subsection{Predicado principal}
\pred{esValido}{t : \toroide} { \\
(\longitud{t} \geq 3 \land esMatriz(t)) \yLuego \longitud{t[0]} \geq 3 \\
}

\subsection{Predicados auxiliares}

\pred{esMatriz}{m: \matriz{T}}{\\
(\forall i : \ent)(0 \leq i < \longitud{m} \implicaLuego (\longitud{m[i]} > 0 \wedge (\forall j : \ent)(0 \leq j < \longitud{m} \implicaLuego \longitud{m[i]} = \longitud{m[j]}))\\
}

\section{Ejercicio 2}

\pred{toroideMuerto}{t : \toroide}{\\
	(\forall i : \ent)((0 \leq i < |t|) \implicaLuego (\forall j : \ent) ((0 \leq j < |t[i]|) \implicaLuego (t[i][j] = \False)))
	\\ }

\section{Ejercicio 3}

\subsection{Función principal}

\pred{posicionesVivas}{t: \toroide, vivas: \TLista{\ent \times \ent}}{\\ 
 (\forall i : \ent)((\forall j : \ent)(0 \leq i < \longitud{t} \yLuego 0 \leq j < \longitud{t[i]})) \implicaLuego ( (i,j) \in vivas \Iff t[i][j] = \True) \land sinRepetidos(vivas) ) \\ 
}

\subsection{Predicados auxiliares}

\pred{sinRepetidos}{vivas : \TLista{\ent \times \ent }  }{ \\
 ( \forall i,j : \ent ) ( (0 \leq i,j < |vivas| \land i \neq j) \Rightarrow_L vivas[i] \neq vivas[j] ) \\
 }



\section{Ejercicio 4}

\subsection{Función principal}
\aux{densidadPoblacion}{t : \toroide}{\float}{\\
	cantidadVivos(t) \ / \ (|t| \cdot |t[0]|)}

\subsection{Funciones auxiliares}
\aux{cantidadVivos}{t : \toroide}{\float}{\\
	\sum_{i=0}^{|t|-1}
	\sum_{j=0}^{|t[i]|-1}
	\IfThenElse{t[i][j] = \True}{1}{0}}

\section{Ejercicio 5}

\subsection{Función principal}
\aux{cantVecinosVivos}{t : \toroide, f : \ent , c : \ent}{\ent}{\\
\sum_{i=0}^{|t|-1} \sum_{j=0}^{|t[i]| -1 } \IfThenElse{t[i][j] = \True \land sonVecinos(i,j,f,c,t)}{1}{0}
}

\subsection{Predicados auxiliares}
\pred{sonVecinos}{x,y : \ent , $v_1, v_2$ : \ent, t : \toroide}{\\
(v_1 \neq x \lor v_2 \neq y) \ \land \ (distancia(x, v_1, |t|) , \ distancia(y, v_2, |t[0]|) \leq 1) \\
} /*Siendo (x,y) las coordenadas de la posición a evaluar y $(v_1, v_2)$ las del vecino */

\subsection{Funciones auxiliares}
\aux{abs}{n : \float}{\float}{ \IfThenElse{n<0}{ -n}{n}}  
\aux{distancia}{$x_1$ : \ent, $x_2$ : \ent, rango: \ent}{\float}{abs (x_1- x_2) \ mod \  rango}
/*Siendo rango la variable para tener en cuenta los vecinos de los bordes*/

\section{Ejercicio 6}

\pred{evolucionDePosicion}{t : \toroide, posicion : $\ent \times \ent$}{\\
	( t[(posicion)_0][(posicion)_1] = \True \land 
	2 \leq cantVecinosVivos(t, (posicion)_0, (posicion)_1) \leq 3 )\\ \lor 
	( t[(posicion)_0][(posicion)_1] = \False \land cantVecinosVivos(t, (posicion)_0, (posicion)_1) = 3 )
	\\}

\section{Ejercicio 7}

\pred{evolucionToroide}{t1: \toroide, t2: \toroide}{\\
(\longitud{t1} = \longitud{t2} \wedge \longitud{t1[0]} = \longitud{t2[0]}) \yLuego \\
(\forall i : \ent)((\forall j : \ent)((0 \leq i < \longitud{t1} \yLuego 0 \leq j < \longitud{t1[i]}) \implicaLuego \\
evolucionDePosicion(t1, (i,j))\Iff t2[i][j] = \True\\
}



\section{Ejercicio 8}

\subsection{Procedimiento}

\begin{proc}{evolucionMultiple}{\In t1 : \toroide, \In k : \ent,  \Out result: \toroide}{}
    \pre{ k \geq 1  \land esValido(t1)}
    
    \post{esValido(result) \land esToroideEvolucionK(k, t1, result)  }
\end{proc}

\subsection{Predicado auxiliar}

\pred{esToroideEvolucionK}{k : \ent , t1, t2 : \toroide}{ \\  k \geq 0 \land_L
( \exists ts : \TLista{\toroide} ) \ ( \ |ts| = k+1 \land  (\forall i : \ent ) (0 \leq i < |ts| -1 \Rightarrow_L ( esValido(ts[i]) \land esValido(t[i+1]) \ ) \land_L  evolucionToroide( ts[i], ts[i+1]) \ )  \land t1 = ts[0] \land t2 = ts[ k ] ) \\
}



\newpage
\section{Ejercicio 9}

\subsection{Procedimiento}
\begin{proc}{esPeriodico}{in t : \toroide, inout p : \ent, out result : \bool}{}
    \pre{(p = P_0) \land esValido(t)}
    \post{\\
    result = \True \Iff esPeriodo(t) \yLuego \\
    (\ (result = \True \Then \neg (\exists k : \ent)(0 < k < p \yLuego esToroideEvolucionK(t, t, k) ) \land \\
    (result = \False \Then p = P_0) \ )\\
    }

\end{proc}

\subsection{Predicados auxiliares}
\begin{pred}{esPeriodo}{t : \toroide}{\\
(\exists k : \ent)(k > 0 \yLuego esToroideEvolucionK(t, t, k))\\
}
\end{pred}


\section{Ejercicio 10}

\subsection{Procedimiento}

\begin{proc}{primosLejanos}{\In t1 : \toroide, \In t2 : \toroide, \Out primos : \bool}{}
    \pre{esValido(t1) \land esValido(t2) \land dim(t1, t2)}
    \post{primos = \True \leftrightarrow sonPrimos(t1,t2)}
\end{proc}

\subsection{Predicados auxiliares}
\pred{dim}{t1, t2 : \toroide}{\\
    |t1| = |t2| \land |t1[0]| = |t2[0]|
    \\}

\pred{sonPrimos}{t1, t2 : \toroide}{\\
    (\exists k : \ent)(k > 0 \land (esToroideEvolucionK(t1, t2, k) \lor esToroideEvolucionK(t2, t1, k)))
    \\}

\section{Ejercicio 11}

\subsection{Procedimiento}
\begin{proc}{seleccionNatural}{\In ts : \TLista{\toroide}, \Out res : \ent}{}
\pre{(\forall i:\ent)(0\leq i < \longitud{ts} \implicaLuego esValido(ts[i])}
\post{(\exists n: \ent)(0\leq n < \longitud{ts} \land \neg(muere(ts[i])) \land res = n)  \vee  (\exists i:\ent)(  0\leq i <\longitud{ts} \land_L \\  (\forall j:\ent)(0\leq j <\longitud{ts} \implicaLuego muere(ts[i]) \yLuego muere(ts[j]) \yLuego muerePrimero(ts[j], ts[i]) ) \land res = i)}
\end{proc}

\subsection{Predicados auxiliares}
\pred{muere}{t : \toroide}{\\
    (\exists tm:\toroide)( esValido(tm) \land_L toroideMuerto(tm) \land sonPrimos(t,tm))\\
}

\pred{muerePrimero}{t1 : \toroide, t2:\toroide}{\\
    (\exists tm:\toroide)( esValido(tm) \land_L toroideMuerto(tm) \land  (\exists n:\ent) \\ (esToroideEvolucionK(n, t1, tm) \land (\forall m:\ent)(esToroideEvolucionK(m, t2, tm) ) \implica n \leq m)))\\
}

\section{Ejercicio 12}

\begin{proc}{fusionar}{\In t1 : \toroide, \In t2 : \toroide, \Out res : \toroide}{}
    \pre{(esValido(t1) \land esValido(t2)) \yLuego dim(t1, t2)}
    \post{esValido(res) \yLuego ( dim(res, t1) \yLuego \\
    (\forall i, j : \ent)((0 \leq i < \longitud{t1}) \land (0 \leq j < \longitud{t1[0]}) \implicaLuego \\
    (res[i][j] = \True \Iff (t1[i][j] = \True \land t2[i][j] = \True) \ ) \ ) \\
    }
\end{proc}    

\section{Ejercicio 13}

\subsection{Procedimiento}

\begin{proc}{vistaTrasladada}{\In t1 : \toroide, \In t2 : \toroide, \Out res : \bool}{}
    \pre{esValido(t1) \land esValido(t2) \land dim(t1, t2)}
    \post{res = \True \leftrightarrow esTraslacion(t1, t2)}
\end{proc}

\subsection{Predicados auxiliares}
\pred{esTraslacion}{t1, t2 : \toroide}{\\
    (\exists x, y : \ent)(\forall i, j : \ent)((0 \leq i < |t1| \land 0 \leq j < |t1[0]|) \implicaLuego \\
    t1[i][j] = t2[i + x \mod |t2|][j + y \mod |t2[0]|])
    \\}

\section{Ejercicio 14}

\subsection{Procedimiento}
\begin{proc}{menorSuperficieViva}{\In t : \toroide, \Out res : \ent}{}
    \pre{esValido(t)}
    \post{   esMenorSuperficie(t,res)}
    
\end{proc}

\subsection{Predicados auxiliares}
\pred{esMenorSuperficie}{t : \toroide, res : \ent}{ \\ 0 \leq res \leq |t|*|t[0]| \land_L   \land  esMenorATodas(t, res) \land  existeEsaSuperficie(t, res)
\\}

\pred{esMenorATodas}{t: \toroide , res : \ent}{ \\ (\forall largo, alto : \ent) ( \ ( \forall i,j : \ent) \ ( \ ( 0 \leq i,alto < |t| \land_L 0 \leq j,largo < |t[i]| ) \\ \Rightarrow_L \ ( \  ( esSuperficieValida(i,j,alto,largo,t) \Rightarrow (alto+1) * (largo+1) \geq res ) \ ) \ ) \ ) \\}

\pred{existeEsaSuperficie}{t : \toroide , res : \ent}{ \\ ( \exists largo_2, alto_2 : \ent) \ ( \exists i,j : \ent) \ ( \ ( 0 \leq i,alto_2 < |t| \land_L 0 \leq j,largo_2 < |t[i]| ) \\ \land_L \ ( \  ( esSuperficieValida(i,j,alto_2,largo_2,t) \Rightarrow (alto_2 +1) * (largo_2+1) = res )) \\} 

\pred{esSuperficieValida}{i,j : \ent , alto , largo : \ent , t : \toroide}{\\ ( \sum_{fi=i}^{i+alto} \sum_{co=j}^{j+largo} \IfThenElse{t[fi \bmod |t|][co \bmod |t[0]|] = true}{1}{0}  ) = cantidadVivos(t) \\}


\section{Ejercicio 15}

\subsection{Procedimiento}
\begin{proc}{enCrecimiento}{\In t : \toroide, \Out res : \bool}{}
    \pre{esValido(t)}
    \post{res = \True \leftrightarrow estaCreciendo(t)}
\end{proc}

\subsection{Predicados auxiliares}
\pred{estaCreciendo}{t : \toroide}{\\
    (\exists t2 : \toroide)( esValido(t2) \land_L evolucionToroide(t, t2) \land \\
    (\exists k, m : \ent)((0 < k < m) \land_L (esMenorSuperficie(t, k) \land esMenorSuperficie(t2, m))))
    \\}

\section{Decisiones tomadas}
En cada ejercicio de la primera parte asumimos que el usuario debe cumplir con el contrato pasando un toroide válido, para simplificar la escritura.
    En general, en los predicados asumimos que los parámetros pasados son válidos cuando posible, y la validez de los parámetros pasados es verificada en los procedimientos.\\ 
\indent En el ejercicio 9, tomamos la decisión de que en caso de que el resultado sea falso, el parámetro p debe tener el mismo valor que se le pasó originalmente. \\
\indent En el ejercicio 10, decidimos que un toroide debe evolucionar una cantidad positiva de ticks. O sea, dado un toroide t, el mismo toroide t 
es primo de t únicamente si t es periódico. \\
\indent En el ejercicio 13, decidimos que la traslación (0, 0), que es igual a la traslación
($|t|$, $|t[0]$) es una traslación válida. 
También, nuestro procedimiento es indiferente a cuál traslación se está evaluando, siempre y cuando sea una traslación válida. 
Asumimos que, por ejemplo, una traslación de (-4, -3) es válida.\\
\indent Además de eso no tomamos ninguna decisión muy relevante al momento de resolver los ejercicios, más que interpretar los enunciados y a partir de eso escribir las especificaciones que pensamos adecuadas.
\end{document}