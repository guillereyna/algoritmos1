\documentclass[a4paper]{article}
\input{Algo1Macros}
\usepackage[utf8]{inputenc}
\usepackage{a4wide}
\usepackage{amsmath, amscd, amssymb, amsthm, latexsym}
\usepackage[spanish,activeacute]{babel}
\usepackage{enumerate}
\usepackage{parskip}

\usepackage{caratula} % Version modificada para usar las macros de algo1 de ~> https://github.com/bcardiff/dc-tex

\newcommand{\toroide}{Toroide}

\setlength{\parindent}{2em} %espaciado horizontal
\setlength{\parskip}{0.5em} %espaciado vertical


\begin{document}

\titulo{TP de Especificación}
\fecha{\today}
\materia{Algoritmos y Estructuras de Datos I}
\grupo{Grupo  3}

% CAMBIAR INTEGRANTES
\integrante{Fernández Spandau, Luciana}{368/20}{fernandezspandau@gmail.com}
\integrante{Reyna Maciel, Guillermo}{393/20}{guille.j.reyna@gmail.com}
\integrante{Casado Farall, Joaquín}{072/20}{joakinfarall@gmail.com}
\integrante{Companeetz, Ezequiel}{415/20}{echucompa@gmail.com}

\maketitle

\section{Ejercicio 1}

\subsection{Predicado principal}
\pred{esValido}{t : \toroide} { \\
(\longitud{t} \geq 3 \land esMatriz(t)) \yLuego \longitud{t[0]} \geq 3 \\
}

\subsection{Predicados auxiliares}

\pred{esMatriz}{m: \matriz{T}}{\\
(\forall i : \ent)(0 \leq i < \longitud{m} \implicaLuego (\longitud{m[i]} > 0 \wedge (\forall j : \ent)(0 \leq j < \longitud{m} \implicaLuego \longitud{m[i]} = \longitud{m[j]}))\\
}

\section{Ejercicio 2}

\pred{toroideMuerto}{t : toroide}{\\
	(\forall i : \ent)((0 \leq i < |t|) \implicaLuego (\forall j : \ent) ((0 \leq j < |t[i]|) \implicaLuego (t[i][j] = \False)))
	\\ }

\section{Ejercicio 3}

\pred{posicionesVivas}{t: \toroide, vivas: \Tlista{\ent \times \ent}}{\\
(\forall i : \ent)((\forall j : \ent)(0 \leq i < \longitud{t} \yLuego 0 \leq j < \longitud{t[i]})) \implicaLuego ( (i,j) \in vivas \Iff t[i][j] = \True)\\
}

\section{Ejercicio 4}

\subsection{Función principal}
\aux{densidadPoblacion}{t : toroide}{\float}{\\
	cantidadVivos(t) \ / \ (|t| \cdot |t[0]|)}

\subsection{Funciones auxiliares}
\aux{cantidadVivos}{t : toroide}{\float}{\\
	\sum_{i=0}^{|t|-1}
	\sum_{j=0}^{|t[i]|-1}
	\IfThenElse{t[i][j] = \True}{1}{0}}

\section{Ejercicio 5}

\subsection{Función principal}
\aux{cantVecinosVivos}{t : \toroide, f : \ent , c : \ent}{\ent}{\\
\sum_{i=0}^{|t|-1} \sum_{j=0}^{|t[i]| -1 } \IfThenElse{t[i][j] = \True \land sonVecinos(i,j,f,c,t)}{1}{0}
}

\subsection{Predicados auxiliares}
\pred{sonVecinos}{x,y : \ent , $v_1, v_2$ : \ent, t : \toroide}{\\
(v_1 \neq x \lor v_2 \neq y) \ \land \ (distancia(x, v_1, |t|) , \ distancia(y, v_2, |t[0]|) \leq 1) \\
} /*Siendo (x,y) las coordenadas de la posición a evaluar y $(v_1, v_2)$ las del vecino */

\subsection{Funciones auxiliares}
\aux{abs}{n : \float}{\float}{ \IfThenElse{n<0}{ -n}{n}}  
\aux{distancia}{$x_1$ : \ent, $x_2$ : \ent, rango: \ent}{\float}{abs (x_1- x_2) \ mod \  rango}
/*Siendo rango la variable para tener en cuenta los vecinos de los bordes*/

\section{Ejercicio 6}

\pred{evolucionDePosicion}{t : toroide, posicion : $\ent \times \ent$}{\\
	\IfThenElse{t[(posicion)_0][(posicion)_1] = \True \\}
	{2 \leq cantVecinosVivos(t, (posicion)_0, (posicion)_1) \leq 3\\}
	{cantVecinosVivos(t, (posicion)_0, (posicion)_1) = 3}
	\\}

\section{Ejercicio 7}

\pred{evolucionToroide}{t1: \toroide, t2: \toroide}{\\
(\longitud{t1} = \longitud{t2} \wedge \longitud{t1[0]} = \longitud{t2[0]}) \yLuego \\
(\forall i : \ent)((\forall j : \ent)((0 \leq i < \longitud{t1} \yLuego 0 \leq j < \longitud{t1[i]}) \implicaLuego \\
evolucionDePosicion(t1, (i,j))\Iff t2[i][j] = \True\\
}

\section{Decisiones tomadas}
En cada ejercicio asumimos que el usuario debe cumplir con el contrato pasando un toroide válido, para simplificar la escritura.
\\Ademas de eso no tomamos ninguna decision muy relevante al momento de resolver los ejercicios, mas que interpretar los enunciados y a partir de eso escribir las especificaciones que pensamos adecuadas.


\end{document}
