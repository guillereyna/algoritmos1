\documentclass[a4paper]{article}
\input{Algo1Macros}
\usepackage[utf8]{inputenc}
\usepackage{a4wide}
\usepackage{amsmath, amscd, amssymb, amsthm, latexsym}
\usepackage[spanish,activeacute]{babel}
\usepackage{enumerate}
\usepackage{parskip}
\usepackage{comment}

\usepackage{caratula} % Version modificada para usar las macros de algo1 de ~> https://github.com/bcardiff/dc-tex

\newcommand{\toroide}{Toroide}

\setlength{\parindent}{2em} %espaciado horizontal
\setlength{\parskip}{0.5em} %espaciado vertical


\begin{document}

\titulo{TP de Especificación}
\fecha{\today}
\materia{Algoritmos y Estructuras de Datos I}
\grupo{Grupo  3}

% CAMBIAR INTEGRANTES
\integrante{Fernández Spandau, Luciana}{368/20}{fernandezspandau@gmail.com}
\integrante{Reyna Maciel, Guillermo}{393/20}{guille.j.reyna@gmail.com}
\integrante{Casado Farall, Joaquín}{072/20}{joakinfarall@gmail.com}
\integrante{Companeetz, Ezequiel}{415/20}{echucompa@gmail.com}

\maketitle



\section{Ejercicio 1}

\subsection{Predicado principal}
\pred{esValido}{t : \toroide} { \\
(\longitud{t} \geq 3 \land esMatriz(t)) \yLuego \longitud{t[0]} \geq 3 \\
}

\subsection{Predicados auxiliares}

\pred{esMatriz}{m: \matriz{T}}{\\
(\forall i : \ent)(0 \leq i < \longitud{m} \implicaLuego (\longitud{m[i]} > 0 \wedge (\forall j : \ent)(0 \leq j < \longitud{m} \implicaLuego \longitud{m[i]} = \longitud{m[j]}))\\
}

\section{Ejercicio 2}

\pred{toroideMuerto}{t : \toroide}{\\
	(\forall i : \ent)((0 \leq i < |t|) \implicaLuego (\forall j : \ent) ((0 \leq j < |t[i]|) \implicaLuego (t[i][j] = \False)))
	\\ }

\section{Ejercicio 3}

\subsection{Predicados auxiliares}

\pred{sinRepetidos}{vivas : \TLista{\ent \times \ent }  }{ \\
 ( \forall i,j : \ent ) ( (0 \leq i,j < |vivas| \land i \neq j) \Rightarrow_L vivas[i] \neq vivas[j] ) \\
 }

\subsection{Función principal}

\pred{posicionesVivas}{t: \toroide, vivas: \TLista{\ent \times \ent}}{\\ 
 (\forall i : \ent)((\forall j : \ent)(0 \leq i < \longitud{t} \yLuego 0 \leq j < \longitud{t[i]})) \implicaLuego ( (i,j) \in vivas \Iff t[i][j] = \True) \land sinRepetidos(vivas) ) \\ 
}

\section{Ejercicio 4}

\subsection{Función principal}
\aux{densidadPoblacion}{t : toroide}{\float}{\\
	cantidadVivos(t) \ / \ (|t| \cdot |t[0]|)}

\subsection{Funciones auxiliares}
\aux{cantidadVivos}{t : toroide}{\float}{\\
	\sum_{i=0}^{|t|-1}
	\sum_{j=0}^{|t[i]|-1}
	\IfThenElse{t[i][j] = \True}{1}{0}}

\section{Ejercicio 5}

\subsection{Función principal}
\aux{cantVecinosVivos}{t : \toroide, f : \ent , c : \ent}{\ent}{\\
\sum_{i=0}^{|t|-1} \sum_{j=0}^{|t[i]| -1 } \IfThenElse{t[i][j] = \True \land sonVecinos(i,j,f,c,t)}{1}{0}
}

\subsection{Predicados auxiliares}
\pred{sonVecinos}{x,y : \ent , $v_1, v_2$ : \ent, t : \toroide}{\\
(v_1 \neq x \lor v_2 \neq y) \ \land \ (distancia(x, v_1, |t|) , \ distancia(y, v_2, |t[0]|) \leq 1) \\
} /*Siendo (x,y) las coordenadas de la posición a evaluar y $(v_1, v_2)$ las del vecino */

\subsection{Funciones auxiliares}
\aux{abs}{n : \float}{\float}{ \IfThenElse{n<0}{ -n}{n}}  
\aux{distancia}{$x_1$ : \ent, $x_2$ : \ent, rango: \ent}{\float}{abs (x_1- x_2) \ mod \  rango}
/*Siendo rango la variable para tener en cuenta los vecinos de los bordes*/

\section{Ejercicio 6}

\pred{evolucionDePosicion}{t : toroide, posicion : $\ent \times \ent$}{\\
	( t[(posicion)_0][(posicion)_1] = \True \land 
	2 \leq cantVecinosVivos(t, (posicion)_0, (posicion)_1) \leq 3 )\\ \lor 
	( t[(posicion)_0][(posicion)_1] = \False \land cantVecinosVivos(t, (posicion)_0, (posicion)_1) = 3 )
	\\}

\section{Ejercicio 7}

\pred{evolucionToroide}{t1: \toroide, t2: \toroide}{\\
(\longitud{t1} = \longitud{t2} \wedge \longitud{t1[0]} = \longitud{t2[0]}) \yLuego \\
(\forall i : \ent)((\forall j : \ent)((0 \leq i < \longitud{t1} \yLuego 0 \leq j < \longitud{t1[i]}) \implicaLuego \\
evolucionDePosicion(t1, (i,j))\Iff t2[i][j] = \True\\
}



\section{Ejercicio 8}

\subsection{Predicado auxiliar}

\pred{esToroideEvolucionK}{k : \ent , t1,t2 : \toroide}{ \\  k \geq 0 \land
( \exists ts : \TLista{\toroide} ) \ ( \ |ts| = k+1 \land  (\forall i : \ent ) (0 \leq i < |ts| -1 \Rightarrow_L esValido(ts[i]) \land esValido(t[i+1]) \land  evolucionToroide( ts[i], ts[i+1]) \ )  \land t1 = ts[0] \land t2 = ts[ |ts| -1 ] ) \\
}

\subsection{Función auxiliar}

\pred{MismaDimension}{t1,t2 : \toroide}{ \\
|t1| = |t2| \land |t2[0]| = |t2[0]| \land esValido(t1) \land esValido(t2)
\\ }

\subsection{Proceso}

\begin{proc}{evolucionMultiple}{\In t1 \toroide, \In k : \ent,  \Out result: \toroide}{}
    \pre{ k \geq 1  \land esValido(t1)}
    
    \post{esValido(res) \land esToroideEvolucionK(k,t1,res)  }
\end{proc}

\section{Ejercicio 9}

\subsection{Predicado Auxiliar} 

\pred{esPeriodo}{ t: \toroide}{ \\
( \exists k: \ent ) \ ( k > 0 \land  esToroideEvolucionK(k,t,t) )
\\ }

\subsection{Proceso}

\begin{proc}{esPeriodico}{ \In t : \toroide, \Inout p : \ent , \Out result : \bool }{}
    \pre{p = P_0 \land esValido(t)}
    
    \post{ result = \True \Leftrightarrow esPeriodo(t) \ \land \\ (result = \True \Rightarrow_L ( \forall k: \ent ) \ ( k > 0 \land  esToroideEvolucionK(k,t,t) \Rightarrow k \leq p )}

%  La segunda condición es para que p sea la menor cantidad de Ticks
\end{proc}

\section{Ejercicio 10}

\subsection{Proceso}

\begin{proc}{primosLejanos}{ \In t1 : \toroide , \In t2 : \toroide,  \Out result : \bool}{}

    \pre{esValido(t1) \land esValido(t2)}
    
    \post{( \exists k: \ent ) \ ( k > 0 \land  ( \ esToroideEvolucionK(k,t1,t2) \ \lor \ esToroideEvolucionK(k,t2,t1) }

\end{proc}

\section{Ejercicio 11}

\subsection{Predicados Auxiliares}

Me dice si el toroide muere en algún momento\\
\pred{Muere}{t : \toroide}{
\neg(esPeriodo(t) \land \neg toroideMuerto(t) )}


/* Es medio inecesario el para todo del tm, pero no sé cómo escribirlo sino. Además está función sólo sirve si el toroide muere, sino siempre da True  */ 

/*Chequea si es la cantidad de ticks hasta morir del toroide */


\pred{esTicksValido}{k : \ent , t : \toroide}{\\
( \forall i: \ent) \ ( \ ( \forall tm :\toroide ) \ ( ( \ toroideMuerto(tm) \land esToroideEvolucionK (i, t1, tm) \ ) \Rightarrow \\
(  esToroideEvolucionK (k , t, tm)  \land i \leq k \ ) \ ) \ ) \land k \geq 0 
\\}

/* Otra forma de pensarlo */ 
Lo podes pensar sino cómo que el toroide la de evolución k-1 está vivo y el de la evolución k la palmo, pero para mi no queda tan bello

/* Devuelve si el toroide está muerto luego de k ticks */

\pred{muertoEnK}{k : \ent , t : \toroide}{\\
( \exists tm : \toroide ) \ ( esValido(tm) \land toroideMuerto(tm) \land esToroideEvolucionK(k,t,tm) \ ) }

\subsection{Proceso}

\begin{proc}{seleccionNatural}{\In ts : \TLista{\toroide} \Out res : \ent}{}
    \pre{( \forall tor : \toroide ) ( tor \in ts \Rightarrow esValido(tor) ) }
    
    \post{ 0 \leq res < |ts| \land ( \neg 
    Muere( ts[res]) \lor  ( \forall i,tick : \ent)  ( 0 \leq i < |ts| \Rightarrow_L ( \ 
    esTickValido(  tick, ts[i] ) \Rightarrow_L \ \neg muertoEnK(tick-1, ts[res]) ) \ ) \ )  
    }

\end{proc}

/* Si hay un toroide que es periodico dentro de la secuencia entonces cómo cualquier tick es válido si hay un toroide que no es periodico no va a poder ser seleccionado ya que va a existir un tick -1 válido en el que muera, por lo tanto sólo los periodicos van a poder ser seleccionados */ 

\section{Ejercicio 12}

\subsection{Funciones Auxiliares}

\aux{min}{x1,x2 : \ent}{\ent}{\IfThenElse{x1<x2}{x1}{x2}}

/* Apariciones de las tuplas */

\aux{apariciones}{vivas : \TLista{\ent \times \ent} , ele : \ent \times \ent}{$\ent$}{ $\sum_{i=0}^{|vivas| -1}$ \IfThenElse{ele = vivas[i]}{1}{0} }

\subsection{Predicados Auxiliares}

\pred{interseccion}{s,t : \toroide, inter : \toroide}{ 
\\ ( \forall i,j : \ent ) ( \ ( 0 \leq i < |s| \land_L 0 \leq j < |s[i]| \ ) \Rightarrow_L ( \ ( ( s[i][j] = \True \land t[i][j] = \True ) \land inter[i][j] = \True ) \lor ( \neg ( s[i][j] = \True \land t[i][j] = \True ) \land inter[i][j] = \False ) \ ) \ )
}

\subsection{Proceso}

\begin{proc}{fusionar}{\In t1 : \toroide, \In t2 : \toroide , \Out res : \toroide}{}
    \pre{mismaDimension(t1,t2)}
    \post{mismaDimension(t1, res) \land interseccion(t1,t2, res)}

\end{proc}


\section{Ejercicio 13}

\subsection{Predicados Auxiliares}

\pred{esTraslacion}{t1,t2 : \toroide}{ mismaDimension(t1, t2) \land \\
( \exists x,y : \ent ) \ ( \ ( \forall i,j : \ent ) \ ( \ ( 0 \leq i < |s| \land_L 0 \leq j < |s[i]| ) \Rightarrow_L t1[i+y \bmod |t| ] \ [j+x \bmod |t[0]| ] = t2[i][j] \ ) )   
}

\begin{proc}{vistaTrasladada}{\In t1 : \toroide, \In t2 : \toroide , \Out res : \bool}{}
    \pre{mismaDimension(t1,t2)}
    \post{res = \True \Leftrightarrow esTraslacion(t1, t2) }

\end{proc}

\section{Ejercicio 14}

\subsection{Predicados auxiliares}

/* Cumple que sea menor igual a todas las superificies posibles e igual a una, por lo tanto es la mínima */

\pred{esMenorSuperficie}{t : \toroide, res : \ent}{ \\ 0 \leq res \leq |t|*|t[0]| \land_L  ( \forall largo, alto : \ent) \ ( \forall i,j : \ent) \ ( \ ( 0 \leq i,alto < 0 \land_L 0 \leq j,largo < |t[i]| ) \\ \Rightarrow_L \ ( \  ( \sum_{fi=i}^{i+alto} \sum_{co=j}^{j+largo} \IfThenElse{t[fi \bmod |t|][co \bmod |t[0]|] = true}{1}{0}  ) = cantidadVivos(t) \Rightarrow alto * largo \leq res ) \land  \\
 ( \exists largo_2, alto_2 : \ent) \ ( \exists i,j : \ent) \ ( \ ( 0 \leq i,alto_2 < 0 \land_L 0 \leq j,largo_2 < |t[i]| ) \\ \Rightarrow_L \ ( \  ( \sum_{fi=i}^{i+alto_2} \sum_{co=j}^{j+largo_2} \IfThenElse{t[fi \bmod |t|][co \bmod |t[0]|] = true}{1}{0}  ) = cantidadVivos(t) \Rightarrow alto_2 * largo_2 = res )) }

\subsection{Procesos}

/* Quedo medio feo, habría que ver cómo usar traslación */

\begin{proc}{menorSuperficieViva}{\In t : \toroide, \Out res : \ent}{}
    \pre{esValido(t)}
    \post{   esMenorSuperficie(t,res)}
    
\end{proc}

\section{Ejercicio 15}

\subsection{Procesos}

\begin{proc}{enCrecimiento}{\In t : \toroide, \Out res : \bool}{}
    \pre{esValido(t)}
    \post{ ( \exists super : \ent ) \ 
    ( \ esMenorSuperficie(t,super) \Rightarrow_L \ ( ( \exists torEvol : \toroide ) ( evolucionToroide(t,torEvol) \land ( \exists superMayor : \ent) ( superMayor > super \land esMenorSuperficie(torEvol,superMayor) \ ) \ ) \ ) \ ) }
    
\end{proc}


\section{Decisiones tomadas}
En cada ejercicio asumimos que el usuario debe cumplir con el contrato pasando un toroide válido, para simplificar la escritura.
\\Ademas de eso no tomamos ninguna decision muy relevante al momento de resolver los ejercicios, mas que interpretar los enunciados y a partir de eso escribir las especificaciones que pensamos adecuadas.

Segunda Parte: 
    Ejercicio 8 + 9: Elegí que un Toroide muerto sea considerado cómo uno periódico de p=1.
    Además si no es periódico p puede ser cualquier cosa y no me importa.
    Ejercicio 11: Cómo los periodicos no se mueren en ninguna cantidad de ticks entonces si hay periodicos en la secuencia res devuelve un periodico

\end{document}
